
\documentclass[10pt]{article}
\linespread{1.25}

%%Make Parenthesis scale to fit whats inside
\newcommand{\parry}[1]{\left( #1 \right)}

%% Language and font encodings
\usepackage[english]{babel}
\usepackage[utf8x]{inputenc}
\usepackage[T1]{fontenc}
\usepackage{subcaption}
\usepackage[section]{placeins}

%% Sets page size and margins
\usepackage[a4paper,top=3cm,bottom=2cm,left=3cm,right=3cm,marginparwidth=1.75cm]{geometry}

%% Useful packages
\usepackage{amsmath}
\usepackage{amssymb}
\usepackage{amsfonts}
\usepackage{mathtools}
\usepackage{graphicx}
\usepackage{xcolor}
\usepackage[colorinlistoftodos]{todonotes}
\usepackage[colorlinks=true, allcolors=blue]{hyperref}
\usepackage{enumerate}
\usepackage{enumitem}
\usepackage{siunitx}
\usepackage{float}
\usepackage{scrextend}
\usepackage[final]{pdfpages}
\usepackage{pythonhighlight}

%%Header & Footer
\usepackage[myheadings]{fullpage}
\usepackage{fancyhdr}
\usepackage{lastpage}
\usepackage{graphicx, wrapfig, subcaption, setspace, booktabs}

%% Define \therefore command
\def\therefore{\boldsymbol{\text{ }
\leavevmode
\lower0.4ex\hbox{$\cdot$}
\kern-.5em\raise0.7ex\hbox{$\cdot$}
\kern-0.55em\lower0.4ex\hbox{$\cdot$}
\thinspace\text{ }}}

%% Units
\DeclareSIUnit\year{yr}
\DeclareSIUnit\dollar{\$}
\DeclareSIUnit\celcius{C^{\circ}}
\DeclareSIUnit\mole{mole}
\def\conclusion{\quad \Rightarrow \quad}

\newcommand{\angled}[1]{\left\langle #1 \right\rangle}
\newcommand{\pd}[1]{\partial_{#1}}
\renewcommand{\vec}[1]{\boldsymbol{#1}}
\newcommand{\M}[1]{\mathbf{#1}}
\newcommand{\grad}{\vec{\nabla}}
\newcommand{\cross}{\vec{\times}}
\newcommand{\laplacian}{\nabla^2}
\makeatletter
\DeclareRobustCommand{\pder}[1]{%
  \@ifnextchar\bgroup{\@pder{#1}}{\@pder{}{#1}}}
\newcommand{\@pder}[2]{\frac{\partial#1}{\partial#2}}
\makeatother
\begin{document}


%----------------------------------------------------------------------------------------
%	TITLE PAGE
%----------------------------------------------------------------------------------------

%----------------------------------------------------------------------------------------
% HEADER AND FOOTER
%----------------------------------------------------------------------------------------
\pagestyle{fancy}
\fancyhf{}
\setlength\headheight{12pt}
\fancyhead[L]{\textbf{MRI Notes}}
\fancyhead[R]{\textbf{Liam O'Connor}}
\fancyfoot[R]{Page \thepage\ of \pageref{LastPage}}

\subsection{Problem Setup}

Consider the non-dissipative incompressible MHD equations in a rotating frame
\begin{align*}
    \partial_t \vec{v} + \vec{v}\cdot\grad\vec{v} + f \vec{\hat{z}}\times\vec{v} + \grad p &= \vec{b}\cdot\grad\vec{b} \\
    \grad\cdot\vec{v} &= 0 \\
    \partial_t \vec{b} + \vec{v}\cdot\grad\vec{b} &= \vec{b}\cdot\grad\vec{v} \\
    \grad\cdot\vec{b} &= 0 \\
\end{align*}

With axisymmetry (constant in $y$) we consider linear perturbations (with $x$ and $z$ dependence of the form)

\begin{align*}
    \vec{v} &= [v_0(x) + v(x,z)]\vec{\hat{y}} - \vec{\hat{y}}\times\grad\psi(x,z) \\
    \vec{b} &= b(x,z)\vec{\hat{y}} - \vec{\hat{y}}\times\grad\left(  a_0(x) + a(x,z)\right) 
    \intertext{such that}
    v_0(x) &= Sx - \frac{Sd}{2} \\
    a_0(x) &= Bx - \frac{Bd}{2}
    \intertext{with $0<x<d$. $S$ is the shear rate of the background azimuthal flow (which decreases in $x$). $B$ is the constant poloidal background magnetic field (in the $z$ direction).}
\end{align*} 

\subsection{Displacement Equation}
The Eulerian displacement vector satisfies
\begin{align*}
  \vec{v} &= \partial_t\vec{\xi} + v_0(x)\vec{\hat{y}}\cdot\grad\vec{\xi} - \vec{\xi} \cdot\grad (v_0(x)\vec{\hat{y}})\\
  &= \partial_t\vec{\xi} - S\xi_x \vec{\hat{y}}
  \intertext{with}
  \vec{b} &= \grad\times(\vec{\xi} \times \vec{B}).
  \intertext{Substituting these expressions into the linearized induction equation gives}
  IND &= \partial_t\vec{b} - \grad\times(\vec{v_0}\times\vec{b} + \vec{v}\times\vec{B})\\
  &= \grad\times(\partial_t\vec{\xi} \times \vec{B}) - \grad\times(\vec{v_0}\times\vec{b} + \partial_t\vec{\xi} \times\vec{B} - S\xi_x \vec{\hat{y}}\times\vec{B})\\
  &= - \grad\times(\vec{v_0}\times\vec{b} - S\xi_x \vec{\hat{y}}\times\vec{B})\\
  &= - \grad\times(Sx\vec{\hat{y}}\times(\grad\times(\vec{\xi} \times \vec{B})) - S\xi_x \vec{\hat{y}}\times\vec{B}).
  \intertext{Notice}
  \grad\times(\vec{\xi} \times \vec{B}) &= \vec{\xi}(\grad \cdot \vec{B}) - \vec{B}(\grad \cdot \vec{\xi}) + (\vec{B} \cdot \grad) \vec{\xi} - (\vec{\xi}\cdot\grad)\vec{B} \\
  &= - \vec{B}(\grad \cdot \vec{\xi}) + (\vec{B} \cdot \grad) \vec{\xi} \\
  &= B\partial_z \vec{\xi} \\
  \intertext{implies}
  IND &= - SB \grad\times(x\vec{\hat{y}}\times \partial_z \vec{\xi} - \xi_x \vec{\hat{y}}\times\vec{\hat{z}}).
  \intertext{However,}
  \grad\times(x\vec{\hat{y}}\times \partial_z \vec{\xi}) &= x\vec{\hat{y}}(\grad\cdot \partial_z \vec{\xi}) - \partial_z \vec{\xi}(\grad\cdot x\vec{\hat{y}}) + (\partial_z \vec{\xi}\cdot\grad)x\vec{\hat{y}} - (x\vec{\hat{y}}\cdot\grad) \partial_z \vec{\xi}\\
  &= \partial_z \xi_x \vec{\hat{y}}
  \intertext{and}
  \grad\times(\xi_x \vec{\hat{y}}\times\vec{\hat{z}}) &= \xi_x \vec{\hat{y}}(\grad\cdot \vec{\hat{z}}) - \vec{\hat{z}}(\grad\cdot \xi_x \vec{\hat{y}}) + (\vec{\hat{z}}\cdot\grad)\xi_x \vec{\hat{y}} - (\xi_x \vec{\hat{y}}\cdot\grad) \vec{\hat{z}}\\
  &= \partial_z\xi_x \vec{\hat{y}}.
  \intertext{Therefore, we automatically satisfy $IND = 0$.}
  \intertext{With the induction equation satisfied, we turn our attention to the momentum equation. Combining the displacement relation with the momentum equation gives the second-order boundary value problem}
  0 &= -\xi_x'' + k_z^2\left( 1 + \frac{\Omega(-B^2k_z^2S+\gamma^2(\Omega - S))}{(B^2k_z^2+\gamma^2)^2} \right)\xi_x.
  \intertext{The growth rate $\gamma\to 0$ for marginal-stability}
  0 &= -\xi_x'' + \left( k_z^2 - \frac{S\Omega}{B^2} \right)\xi_x.
  \intertext{Thus, for finite $k_z$, we obtain a marginally-stable mode when}
  \sqrt{\frac{S\Omega}{B^2}-k_z^2} &= \frac{d}{\pi}.
  \intertext{We define the potential}
  V &\equiv -\frac{S\Omega}{B^2}.
  \intertext{Next we multiply the $x$-displacement equation by $\xi_x^*$ and integrate over the domain, yielding a functional}
  \mathcal{F} &\equiv \int_0^d  |\xi_x'|^2 + k_z^2|\xi_x|^2 + V|\xi_x|^2 dx.
  \intertext{The displacement $\xi_x = A \sin (\pi x / d)$}
  \mathcal{F} &= A^2\left(\frac{\pi^2}{2d} + k_z^2\frac{d}{2} + V\frac{d}{2}\right).
  \intertext{Saturation occurs when}
  \mathcal{F} &= -\delta\mathcal{F}
  \intertext{where}
  \delta\mathcal{F} &= \int_0^d \delta V|\xi_x|^2 dx.
  \intertext{The potential varies due to changes in the background magnetic field and shear parameter, i.e.}
  \delta V &= -\Omega B^{-2}\delta S + 2S\Omega B^{-3} \delta B
\end{align*}

\subsection{Eigenvalue Problem}

\begin{align*} 
    \intertext{Substitution yields}
    \partial_t b + J(\psi, b) &= J(a, v)+B\partial_zv - S\partial_za , \\
    \partial_t a + J(\psi, a) &= B\partial_z\psi , \\
    \partial_t v + J(\psi, v) - (f+S)\partial_z\psi &= J(a,b) + B\partial_zb , \\
    \partial_t\laplacian \psi + J(\psi, \laplacian \psi) + f\partial_zv &= J(a,\laplacian a) + B\partial_z\laplacian a 
    \intertext{where the Jacobian}
    J(p,q) &\equiv \partial_xp\partial_zq - \partial_zp\partial_xq
    \intertext{We denote linear perturbations with apostrophes, e.g. $v(x,z,t)=v_0(x) + v'(x,z,t)$. We discard the Jacobian terms to obtain the linearized system
    }
    \partial_t b' &= B\partial_zv' - S\partial_za' , \\
    \partial_t a' &= B\partial_z\psi', \\
    \partial_t v' - (f+S)\partial_z\psi' &= B\partial_zb', \\
    \partial_t\laplacian \psi' + f\partial_zv' &= B\partial_z\laplacian a'.
    \intertext{We isolate the nonlinear feedback by taking the $z$-average of the nonlinear equations. For velocity, we rewrite the $y$-component equation as}
    \partial_t \angled{v_y}  &= \angled{\vec{b}\cdot\grad b_y - \vec{v}\cdot\grad v_y} \\
    &= \angled{\grad\cdot (b_y\vec{b} - v_y\vec{v})} \\
    &= \partial_x\angled{-b_y\partial_z a + v_y\partial_z\psi}.
    \intertext{For the induction equation, we use the scalar potential}
    \partial_t \angled{a} &= -\angled{J(\psi, a)} \\
    &= \angled{\partial_z\psi\partial_x a - \partial_x\psi\partial_z a} \\
    &= -\angled{u_x\partial_x a + u_z\partial_z a} \\ 
    &= -\angled{u_x\partial_x a - a\partial_z u_z}.
    \intertext{The velocity has no divergence and no $y$-dependence, meaning $\partial_x u_x+\partial_z u_z=0$ such that}
    &= -\angled{u_x\partial_x a + a\partial_x u_x} \\
    &= -\partial_x \angled{a u_x} \\
    &= \partial_x \angled{a \partial_z\psi} \\
    % \partial_t \angled{\vec{b}} &= \angled{\vec{b}\cdot\grad\vec{v} - \vec{v}\cdot\grad\vec{b}}. \\
    \intertext{The background state is characterized by the shear parameter $S$ as well as the vertical magnetic field $B$. We eliminate the time derivatives by manipulating the linear equations. For the velocity, we have}
    \partial_t \angled{v_y} &= \partial_x\angled{-b_y\partial_z a + B^{-1}v_y\partial_t a} \\
    &= \partial_x\angled{a\partial_z b_y + B^{-1}v_y\partial_t a} \\
    &= B^{-1}\partial_x\angled{a \partial_tv_y - (f + S)a \partial_z \psi + v_y\partial_t a} \\
    &= B^{-1}\partial_x\angled{a \partial_tv_y - \frac{f + S}{B} a \partial_t a + v_y\partial_t a} \\
    &= B^{-1}\partial_x\angled{\partial_t\left( av_y \right) - \frac{f + S}{2B}  \partial_t \left( a^2 \right)} \\
    &= \partial_t\left(B^{-1}\partial_x\angled{ av_y - \frac{f + S}{2B} a^2 }\right) \\
    \therefore \; \angled{v_y} &= Sx + \partial_x\angled{ \frac{1}{B}av_y - \frac{f + S}{2B^2} a^2 }
    \intertext{For the magnetic potential, we have}
    \partial_t \angled{a} &= B^{-1}\partial_x \angled{a \partial_t a} \\
    &= \partial_t\left(\frac{1}{2B}\partial_x \angled{a^2} \right) \\
    \therefore \; \angled{a} &= Bx + \frac{1}{2B}\partial_x \angled{a^2} \\
\end{align*}

\subsection{Amplitude Estimate}
The feedback terms from the previous subsection allow us to compute the variations of the background quantities
\begin{align*}
  \delta S(x) &= \partial_x^2\angled{ \frac{1}{B}av_y - \frac{f + S}{2B^2} a^2 } \\
  \delta B(x) &= \frac{1}{2B}\partial_x^2 \angled{a^2} 
  \intertext{Once the $x$-displacement $\xi_x$ is known, we can compute the other eigenfunctions}
  a &= -B\xi_x \\
  v_y &= \frac{\left( \gamma^2k_z^{-1} - B^2k_z \right) \left( k_z^2 - \pi^2 d^{-2} \right) }{2\Omega k_z}\xi_x = \kappa \xi_x.
  \intertext{Therefore}
  \delta S(x) &= \partial_x^2\angled{ \frac{1}{B}av_y - \frac{2\Omega + S}{2B^2} a^2 } \\
  \delta B(x) &= \frac{1}{2B}\partial_x^2 \angled{a^2} 
  \intertext{Recall,}
  \delta V &= -\Omega B^{-2}\delta S + 2S\Omega B^{-3} \delta B \\
  &= -\Omega B^{-2}\partial_x^2\angled{ \frac{1}{B}av_y - \frac{2\Omega + S}{2B^2} a^2 } + S\Omega B^{-4} \partial_x^2 \angled{a^2}.
  \intertext{Then, writing everything in terms of the $x$-displacement}
  &= \Omega B^{-2}\partial_x^2\angled{ \kappa \xi_x^2 + \frac{2\Omega + S}{2} \xi_x^2 } + S\Omega B^{-2} \partial_x^2 \angled{\xi_x^2} \\
  &= \Omega B^{-2}\left(\kappa + \Omega + \frac{3S}{2}\right)\partial_x^2\angled{ \xi_x^2 }
  \intertext{implies}
  \delta\mathcal{F} &= \int_0^d |\xi_x|^2(\Omega B^{-2}\left(\kappa + \Omega + \frac{3S}{2}\right)\partial_x^2\angled{ \xi_x^2 }) dx \\
  &= A^4 \frac{3\pi^2}{8d}\Omega B^{-2}\left(\kappa + \Omega + \frac{3S}{2}\right).
  \intertext{Therefore, the overall amplitude estimate}
  A &= \sqrt{\frac{\left(\frac{\pi^2}{2d} + k_z^2\frac{d}{2} + V\frac{d}{2}\right)}{\frac{3\pi^2}{8d}\Omega B^{-2}\left(\kappa + \Omega + \frac{3S}{2}\right)}}
\end{align*}

\end{document}
